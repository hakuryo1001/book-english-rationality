\chapter{}
 󱝚 \ruby{}{The Rule of Law} 󰳙\lr{黹}{黹},實在 係 一個 󱭱󱸙󱏡 得意 盞鬼 󱝚 一個 。「」,監生 翻譯 嘅 󱁚係 「法治」,但係 如果 要 傳神啲, 要 原汁原味啲, 要 捕捉到 「」 裏面 啲 組成成分 之間 透過住 啲 介詞 所 散溢 

The concept of “the rule of law”, is just so fascinating a concept. There is so much one can unpack from it via a linguistic-philosophical viewpoint. The use of the definite article, the choice of the preposition “of”, the question of what the signifant “law” is signifying and referring to, the question of what constitutes “rule”... these are important and very interesting questions, but admittedly questions that are possibly philosophically methodologically flawed... I shall come back to that later one day. 

「法治」呢個概念,真係迷人到不得了。從語言哲學嘅角度去拆解,可以講到花都謝。個「the」點解要用定冠詞?「of」呢個介詞點解咁揀?「law」呢個能指,到底係指緊乜嘢?「rule」又究竟係由乜嘢構成?呢啲都係重要而且非常有趣嘅問題。但講真,呢啲問題可能本身就有哲學方法論上嘅漏洞同缺陷。

\section{}
The Rule of Law, as described by the great English jurists, the kind quoted by John Adams from James Harrington in "We are nation of Laws, not of Men", is one of the greatest products of invention and evolution. But it is fundamentally a myth, an impossibility, a square circle, a paradox. 

If the Rule of Law, over a nation, is fundamentally different from that of the Rule of Men, then it must be saying that the rule and operation of law is beyond and above the orbits of politics - which is what the relations of men is about.

So conceived, the rule of law asserts itself to be above, parallel, indifferent to matters of politics, of political opinion wherever one may find them. It asserts to deliver justice and justice it delivers. Alas, it even asserts that justice is in of itself unrelated to matters of politics. 

How does the rule of law then promise the delivery thereof? It asserts that through the deployment of Reason, through Logic, through a series of just procedures (that are also discoverable through Reason), that justice may be begotten and delivered. Justice, in the eyes of the rule of law, is tautology. It needs no premises. It can be created through the machinery of procedure. In other words, procedural justice can beget substantive justice. Yet, on the other hand, the rule of law also asserts that it is not the rule of the law of man that is at work. It asserts that it is the law of god, the law of the celestial that is at work. The law of man is politics. The law of god, that’s morality. 

Morality can be discovered through reason, and it is the rule of The LAW that the rule of law promises. Here lies the problem. The questions of morality,is fundamentally isomorphic to the question Of the questions of politics. The questions of morality ask us to adjudicate right from wrong. Politics also ask us to adjudicate right from wrong. But moreover, the assertion that justice can be begotten procedurally, from no premises, or from other tautologies, from so-called self-evident truths, is a myth, a lie, and a dogma. The self-evidentiality of those premises, stem from that very same source that which our political and moral self-evident first principles stem from. From the same constitution are two breasts with milk flow. The rule of law, claims it is agnostic to matters of politics, and is only predisposed to judging matters of morality - yet the two are one in the same. The rule of law, is a myth. 


所謂 法治,英國啲大法學家講到好巴閉,仲畀約翰亞當斯搬過去引述詹士哈靈頓嗰句:「我哋係一個法律嘅國家,而唔係人治嘅國家。」聽落去叻到爆,其實係人類發明同演化嘅偉大產物之一。但係,掀開塊面紗一睇,原來根本係神話,係唔可能嘅嘢,好似 圓󱝚 正方形 󱜩󱝚 矛盾。

點解?如果「法治」真係同「人治」係兩回事,即係話法律嘅運作,係凌駕喺政治之上——但政治唔就係人同人之間嘅角力咩?

咁樣構思落嚟,法治就話自己高高在上,凌駕於政治之上,對政見冷冷清清,無動於衷。佢話自己交付嘅係「正義」,就係「正義」。仲要講到口響響:正義根本同政治無關。

咁問題嚟啦,法治點樣交貨?佢話靠理性,靠邏輯,靠一套「公正程序」——而呢啲程序又可以透過理性嚟搵到。咁樣,正義就會自自然然咁生出嚟。喺法治嘅眼中,正義係套 tautology——即係自圓其說,唔使前設,靠程序就可以孵出實質嘅正義。

但轉個頭,法治又話,運作嘅唔係人間之法,而係「天道」、「神之法」。人間之法,就係政治;神之法,就係道德。

佢仲講,話道德可以靠理性去發現,而法治承諾嘅,就係呢套「大寫嘅 LAW」。問題就係:道德嘅問題,本質上同政治嘅問題係一樣。道德要我哋判斷「對與錯」,政治亦都係要判斷「對與錯」。更加大鑊嘅係,話「正義」可以靠程序、靠自明真理、靠空中樓閣嘅 tautology 生出嚟——其實都係呃仔呃女。所謂「自明」嘅真理,根本同政治、同道德嘅第一原則,一樣係同一個奶頭流出嚟嘅奶水。

法治成日話自己對政治唔偏不倚,淨係裁決道德。但講到尾,道德同政治係一回事。法治,本身就係一個神話。

\section{}


It is probably not a mischaracterisation to say that the west's conception of freedom and liberty is a dogmatic one. The value of freedom and liberty is not justified via anything or by anything but is taken to be an inherent good. And it is taken to be the good for no heuristic reason. Modern Chinese civilisation has absorbed and assimilated the language of freedom and liberty but reinterpreted it through a paradigm of utilities. Chinese freedom is therefore the product of a kind of rule utilitarianism. 

Through this (and many other places) we see a recurring theme that characterised the difference between Chinese and western thought and philosophy. This has been almost unanimously acknowledged by philosophers and intellectuals since the Great  Bitchslap on the Chinese intellectual face by the western hand of the late Qing and early republican era. Though the discussion has only scratched the surface of the issue.

There is almost no disagreement to the claim that Chinese philosophy is almost thoroughly characterised by its predisposition towards "use" 用— practical application. Chinese philosophy starts and ends with the use of the objects and concepts discussed. They are to serve a function. Abstractions, few and far in between as a result of this dogma seeping into the realm of methodology as well, if used, are only to serve the purpose of elucidating concepts that are to be used and used in a substantially practical sense. So even the Song neo-Confucians' talk of metaphysics are ultimately constructed to justify the Confucian institutions so they may continue to be legitimately used in society. 

But this concern with practical application is not just a mere obsession, or rather - it is a not an obsession that lies at the root of it all, it is but only a manifestation, albeit its most obvious and recurring manifestation of a deeper dogma. There is a myriad of other phenomena that are also manifestations of that dogma. The following are that which I consider to be noteworthy examples: 
* is-ought confusions in the classics but also modern Chinese philosophy and popular every day intellectual discourse. (The kind of talk along the lines of 「方法就係唔啱」commonly seen during occupy central) 
* The absence of the development of abstract axiomatic mathematics 
* The absence of the development of logic 
* Absence of the rule of law in favour of the rule of man 
* High adherence and widespread adherence of utilitariansm in china 
* Under-emphasis of procedural justice 
* No conception of rights, liberty, freedom. A questionable conception of justice. 
* Underdevelopment of metaphysics. 
* Underdevelopment of music, visual arts, and literature 
* * The reach of the bureaucracy in not just Chinese politics but also into the afterlife in Chinese mythology

That dogma, is one that is rooted in realism - realism not in the sense of the word in international relations or in the sense of the word of "pragmatism" but in the following sense: the Chinese are thoroughly realists in the sense that they do are concerned with this world and this world only. Modal propositions (be it demotic, epistemic, etc) begin with our world, and end with a world that is accessible. Normative propositions are only concerned with worlds that are accessible - and accessible worlds are only accessible if and only if they are realistic. Chinese philosophy is realist in that sense. Pragmatism of politics and personal life choices and realism of international relations are but only particular instantiations of this deeper dogma. 

This is not to say that Chinese philosophy is not idealistic, nor does it imply that Chinese philosophy is non-metaphysical. If we are to understand idealism in the sense that a proposition is idealistic if it is modal (normative especially), then certainly it is absolutely false that Chinese philosophy is not idealistic. Verily, the entirety of the Confucian project is to realise their ideal political order! Ren 仁 is an ideal! To follow the Dao 到 is also an ideal! To love and care impartially 兼愛 is an ideal! 君君臣臣父父子子 is an ideal! 

That's obvious. But what about metaphysics? How can we adequately respond to Hegel's vicious and infamous charge that China has no philosophy (for she has no metaphysics)? This calls for some simple characterisation of metphayiscs, a wuasi definition of the subject (the kind that Nietzsche abhors). What is metaphysics? It is meta-physics. Above physics. Beyond physics. Behind physics. 形而上。above our earthly forms. What is. Not that seems to be is, but that really is is. In this account then, though it is probably that the Chinese metaphysics discussion is nowhere as sophisticated or diversified as the western tradition, it would be still be false to claim that china is devoid of metphayiscs. Putting aside the obvious Neo-confucians who themselves coined the Chinese term of metaphysics 形而上學, let us simply talk about the Dao 道。the Dao is a thoroughly metaphysical concept, notwithstanding the natural, earthly connotations implicated by its sinoglyph. We can see it's metaphyidicality by its substance.  
What is the Dao? The Dao is the order that governs all things. It is but it is also more than the mere laws of physics. It is the reason in which existence governs and has come to govern itself. It is that which governs how this world evolves into another accessible world. It is that which is and that which ought to be. From this, it's metaphysicaloty is obvious. The Dao is as metaphysical as Tian is, as God is. 

So Chinese philosophy may be characterised as a form of metphayiscs realistic idealism. 








This paradigm which is throughly soaked in use-soup Which is why utilitiaria





I have just read a comic characterising the reasoning structure of the philosopher as follows: 

If p is true then X. 
If X, then I will be sad. 
I do not wish to be sad. 
Therefore not X, and therefore p is false. 

This is true. When we deconstruct opposing theories, the fundamental clash we shall see stems from a difference of values or maxims we take as axioms or self-evident truths. The west values democracy, equality, this and that, ultimately because it takes the human being to be the Good. And the reason it takes the human to be the Good is because God loves all men equally (etc...), i.e.: if not man as the Good, they will be sad. 

The tragedy is not so much our inability to ground our philosophy in absolutely self-evident truths, but in that we have no way to be sure or Goods are correct and that we have no way to adjudicate different values. The forbidden fruit has given us the ability to tell good from evil in one paradigm - it has not given us the ability to tell if that paradigm is correct. That ability lies still and only with God, and for that, you can still hear God's sniggering at our stupidity. 



Why are all laws fundamentally legislation? 
Possible answers
1. All reasoning fundamentally involves choice making. Choice making is legislation. 
2. The process of oracle guessing is judge law making. law making is legislation. The metric is whether the legislation is to compare with natural law. 
3. Law making cannot make sense without l


The second amendment is rooted in two sources of constitutional traditions, one ancient and Anglo Saxon, the other modern, enlighenment, and French-Swiss. The most explicit reference to the right to bear arms in English constitutionalism can be found in the English Hill of Rights of 1689, passed in the aftermath of the Glorious Revolution which deposed the Catholic King James II, allowing Protestant subjects of England and Wales to 
"have Arms for their Defence suitable to their Conditions and as allowed by Law." The French-Swiss continental source stems from the highly autonomous cantons of the Swiss, which inspired the French writers of the enlightenment on matters of constitutional design. Rousseau was born Swiss. The right to bear arms in the enlightenment tradition is seen not just as a necessary check against monarchical or presidential powers that threaten to become monarchical via their power to wield standing armies, but an indivisible component of what constitutes a citizen - a soldier. In the enlightenment perspective, what republics are constituted of are citizen-soldiers, ready to form militias to defend and die for the republic, as romanticised in republican hymns and anthems such as 血染的風采,  battle hymn of the republic, march of the volunteers, les Marseilles. This enlightenment strand is also the dangerous thread connecting enlightenment’s democracy with fascism - not it’s not the actual right to bear arms but the democratic proposal that the people should be the source of political legitimacy that’s responsible for fascism. Political power is poisonous. 




And this will corrupt to the British common law system. The common law system is no protection for human rights. It appears to do so only because the aim of British civilisation is so. When applied to authoritarianism common law with faithfully and unfailingly carry out its duty. Common law might be the closest thing to the Rule of God. But it is still the rule of man. 



【Amnesty: On the Rule of Law】

Ever since Umbrella, the Hong Kong public’s understanding of what the rule of law is has been continuously under assault by the government’s philistine and intellectually and legally disingenuous nonsensical reinterpretation of the term – to the point that some have already forgotten its very concept. The current government response to the protests, and the litany of apologies and ludicrous defences to excuse the police of any wrongdoing and law-breaking, has not only further polluted the public conception of the rule of law – it is currently mutating the public institutions that manifest the very rule of law itself.
 
The orthodox elucidation of the above would often involve a clear list of the most egregious instances of governmental attempt on the rule of law. Little attention is given to the role of language, and the intellectually disingenuous use thereof. In particular, the different idiosyncrasies of the Chinese writing system and the Cantonese language are here somewhat to blame to make this bastardisation process so easy. The term “rule of law” might have already been naturalised into Sinitic vocabulary, but the concept is still mostly alien. More importantly, the term still lives its life very much under the influence and shadow of traditional Sinitic vocabulary, and therefore traditional Confucian-Legalist thought.
 
The Chinese translation of the term “Rule of Law ” is faat3 zi6 法治. The term is composed of faat 法 “law ” and zi 治 . Zi 治 here roughly means “to rule ” as in gun2 zi6 管治 “govern” but it also carries strong connotations of “to cure ” and “to rectify” as in ji1 zi6 醫治 “to cure” and zi6 liu4 治療 “to medically treat”, and “order ” as in zi6 lei5 治理 “to govern”, daai6 zi6 大治 “great order / peace”, zi6 ngon1 治安 “security”. This zi, which carries connotations of “to cure”, “to rectify” and to “bring about order” is perhaps best exemplified in the term zing3 zi6 政治 “politics” and zing2 zi6 整治 “to mess with someone until they are no longer an obstacle”. (How is that not precisely what the government and the police are doing with the protestors with baton on their right hand and the law on their left?)
 
One would also note that there is no preposition to clarify the exact relationship of faat and zi. Furthermore, if zi is understood to be “order ”, then a state of zi is clearly the opposite to its antonym lyun6 亂 “disorder”. Thus , what we have here , is a subtle and sneaky substitution of concepts exploiting the linguistic characteristics of Chinese characters and the Standard Written Vernacular. Some say that the concept of faat zi propelled by the Hong Kong government is “rule by law ” but not “rule of law ”. That is true but not the full picture. In this bastardised and siniticised conception of faat zi , the existence of faat zi is to achieve order, to cure society of disorder , through the use of law. Thus faat zi 法治 is merely a complement to zing zi政治, the latter attempts to zi, to bring about order through zhing “politics”, “policies” or “political means” – the former attempts the very same but through faat “law”.
 
 
Naturally, in this conception of faat zi, there is no need for moral content or justification for the law. And conveniently, neither Cantonese nor the standard vernaculars encourage its users to question the moral content or justification of laws. It does not matter whether the law is just or justified or even justifiable. All that matters is that it can be used to achieve order. Thus, protestors, and indeed any socially destabilizing elements, are a threat to faat zi, the rule, of law. In other words, breaking the law, is identical to assaulting the law’s rule. Never has it occurred to them that laws are meant to be broken and under their logic there could be no faat zi as soon as anybody breaks any law.
 
What is subsumed underneath all of this false jurisprudence, is of course the authentic rule of law – which like all things British, must be swept away in this process of so-called decolonization a ala chinoise. What is this concept? It is the deep embodiment of and commitment to natural law, and to the freedoms, the rights, and the pursuit and delivery of justice part and parcel of the common law system. In this mutation of the common law system, there is no such thing as “Higher Law” in its legal ontology. Thus, the “rule of Law” in the government perspective, is merely the preservation of the framework of the common law system, all its logic and precedent-calculus, but none of its substantive commitment to justice. And in this system, to withstand the most ancient and common sense of legal maxims, Les iniusta non est lex, that “an unjust law is no law at all”, the government must therefore advance an entirely formalistic and procedural conception of the rule of law. The government sells the idea, that the operation of the law is just because it has fulfilled all the procedures with formulaic accuracy. They must claim, that justice stems from the procedure, from the formulae, and that only. There is no substantive justice. All justice is to be found within the calculus of the judiciary.
 
Thus the government works tirelessly everyday to churn out parochialisms, platitudes, and plain poppycock to confound the common man, to confuse the difference between the law as it is and the law as it should be, or worse, the law as it used to be and the law as they made it up. There could be no justice through illegality, they proclaimeth, and all that is just is that which is legal and vice versa – as so many spokesman from the Chinese Liaison Office and the Hong Kong and Macau Affairs Office have said so many times – notwithstanding the fact that none of them have the slightest training in English jurisprudence.
 
But of course, as so many condemning the protestors would be all too eager to point out, that many of the laws broken are not themselves unjust: and the protestors should all be held legally responsible for their actions. “Crimes!” they cry. Crimes like obstructing public traffic, occupying of public roads, assaulting police officers, damage of public property, arson even! And what injustice is there in those laws that proclaim such actions to be crimes?
 
No, of course, those laws are not in themselves unjust. But this is simple mis-intellectualisation. A law in itself does not carry justice: it is in its application and its deployment that justice and injustice are produce. A well-intended law, misapplied, or applied in circumstances unreasonable for its application, begets injustice. For example, if the effects procured from an application of a law are disproportionate, or out of place, or simply ludicrous to the circumstances, then the application of the law, is unjust. For example, finding a woman who has been indecently assaulted by a policemen guilty of police assault, is precisely the kind of ludicrous application of the law to unjust ends. The kind of prosecution of the attacked instead of the attacker now so a favourite method by the police to harass protestors or their sympathisers, is another.
 
 
Quod necessarium est lictium. “That which is necessary is legal.” This is another way of saying what is moral must be legal: in other words, if something is morally permissible, or morally obligatory, then the law must not punish it.
 
This is where the divorce of the legal (and therefore the judicial) and the political enforced by the separation of powers or “judicial independence” collapses at a philosophical level. The courts might be independent from parliament (or in our case, our undemocratic and barely functional legislature composed of mostly pro-Beijing clowns known as Legco) in structure and functionality, but deep down fundamentally, the decisions made by the court are never purely legal. They contain an ounce of the political. If the judiciary is where the legal is decided, politics is definitely the arena where the thousand public conceptions of what is right and what is good deliberate to produce “policies” and “law”. And what is morality, what is justice, if not those profound principles that inform what is right and what is good? The legal and the political, are therefore fundamentally entwined. To ignore this, is to pretend that the legal system through its mechanical application, without regard for its legitimacy and moral content, can produce justice by itself. It is to defenestrate substantive justice in favour of formalistic and procedural justice. And that is what the government, both Hong Kong and Chinese, have increasingly made clear to be their way of doing things: ji1 faat3 baan6 si6 依法辦事 “[we are] doing things according to the law”.
 
This means that law-breaking, in the form of violent protests and riots, from the protestors, must all be just, if they are necessary. Therefore, any administration of punishment thereof, however legally sanctioned and legally justified, would be most unjust. The law applied will then be an unjustly applied law, which is to say no application of law at all. It is tyranny; it is brutality; and it is evil.
 
How do we know such acts are necessary? 30 years of democratic deficit, government-corporate collusion, white elephant projects, mismanagement and outdated policies in housing, education, and healthcare, purposeful destruction of local language, culture, and customs, continued frustration of democratic will be it by disqualification of democratically elected representatives, capricious reinterpretations of the Basic Law, or political prosecutions and preposterous charges cooked up to harass citizens and activists, the government has done absolutely nothing to alleviate the situation except to continue lining up government cronies and tycoons with goodies. There is no way out, and if that meant breaking a few MTR stations to escape this subterranean heat, then any law finding such violence punishable must be unjust.
 
The ultimate injustice, would be to in any way punish the arrested protestors. The political and the moral (the just!) demand the legal to stand aside. The cause of the protestors, is just, and everybody knows this. Everybody in the government in the right mind knows this. Everybody knows this has to be true. Anybody mildly educated, who has heard of Malcolm X, must known that without these violent actions, justice would never granted, and freedom would never arrive. Freedom is not free.
 
Passionate might be the political and the moral impulse of man, and therefore perhaps much less tempered and consistent than the cold, precedent-bound legal system: but it is no less rational. Legal judgments are rational because they are legally computationally rationale, whereas moral and political judgments are rational because they are deontically and epistemically rational and truthful. What we have here, is a city whose socio-economic environment and political arrangement have produced necessary illegality in the pursuit of great justice and the prevention of great evil. To let the legal machinery run without regard of that under the excuse of “the rule of law” would be actively enable injustice.
 
There is no doubt. There is no question. Prosecutions of protestors must cease. There must be amnesty. There is a reason why that is a demand amongst the five. And there is a reason why none of the five are negotiable.
 
 
 
 Natural law exists as a moral algorithm. All normative questions of politics are fundamentally indiguishable from moral questions. The aim of society, through its legal system and legislature, is to realise the moral algorithm, natural law. In an imaginary world where the moral algorithm is indeed realised through society perfectly, one must follow the societal law without fault, for in this world, the societal law IS the natural law, and the natural law is infalliable.But this aim and this aspiration, by its very nature, must always fall short. No societal law can ever capture the natural law in perfection. This means obligation to follow societal law, is never complete or absolute. To the contrary, this discrepancy implies the individual always retains that right to disobedience, rebellion, and violence. When faced with injustice, if there is no recourse in the societal system, this right to violence is ever more accentuated. But alas, even in situations where there IS recourse, this right remains undiminished, and is ever more legitimately exerciseable. For exercise of this right, is the only reason for that society to be so afraid that they shall act to prevent these situations of injustice from happening again. A society must always fall short of the just world, but without reminders of the horrors of the state of nature, it may stagnate, and will indeed have plenty of reasons of stagnate. Let the blood stench of the state of nature propel society towards justice evermore. 




所以 over my dead body. 
Any claim that the rule of law in hk survives is based on semantics and a naive understanding of the rule of law. We as hongkongers of course are thoroughly brainwashed by the British rule of law tradition: its promises and its illusions. I will never forget Alvin Yeung’s reaction when I said directly to his face 我覺得你對中國嘅理解好俍。We have inherited the British’s framework on the rule of law, but we have not inherited its essence, which is - as much as modern leftist cosmopolitans try to claim otherwise - based on fundamentally a judeo Christian value system, metaphysics, and ontology - Natural Law. I have never understood why could people adhere to legal positivism. It is untenable a position. Even if I grant you on surface levels that laws are positivist if they are not ultimately grounded on the assumption that there exists a natural law system, how on earth can the law derive its moral force? If it can’t, then it is just sheer naked power. Then it has absolutely nothing to do with justice but everything to do with forceful and violent solicitation of obedience - which is of course, what he Hk legal system is doing. Judges are not saints. They’re people. And in the case of Hong Kong, they are CHINESE before they are people. 






"It is true that the theory of our Constitution is, that all taxes are paid voluntarily; that our government is a mutual insurance company, voluntarily entered into by the people with each other; that each man makes a free and purely voluntary contract with all others who are parties to the Constitution, to pay so much money for so much protection, the same as he does with any other insurance company; and that he is just as free not to be protected, and not to pay tax, as he is to pay a tax, and be protected. But this theory of our government is wholly different from the practical fact. The fact is that the government, like a highwayman, says to a man: “Your money, or your life.” And many, if not most, taxes are paid under the compulsion of that threat. The government does not, indeed, waylay a man in a lonely place, spring upon him from the roadside, and, holding a pistol to his head, proceed to rifle his pockets. But the robbery is none the less a robbery on that account; and it is far more dastardly and shameful. "[4]


Where and how does a representative government acquire authority to perform such coercive acts, acts disallowed for individual citizens?”[

the people cannot delegate to government the power to do anything which would be unlawful for them to do themselves.[6]



"No man has a natural right to commit aggression on the equal rights of another, and this is all from which the laws ought to restrain him.”


“The Parliament of Great Britain hath no more Right to put their hands into my Pocket without my consent, than I have to put my hands into yours, for money.1



第一,法律無非是習慣的總結。是不是普世的,很成問題。希羅多德就是這種觀點。印度人把他們的父親在死後燒成灰,埃及人把他們的父親在死後做成木乃伊而永久保存下來。而希臘人和波斯人介於兩者之間,把他們的死人埋葬起來。

誰是最終的正義?TMD,一切都是習俗而已,無非就是習俗,法律就是一種比較精煉的習俗。這是一種觀點。另一種觀點是,法律是普遍的正義,體現了柏拉圖原型式的理性。

但是我們可以利用完美的幾何——歐幾裏德幾何作為模型,來指導和掌控現實世界。法律也是這樣。世界上存在著唯一的真理和正義,它體現於理性的法律。現實上被希羅多德一派認為不過是習俗的東西,就像是張三建築師和李四律師畫出來的直線全都是不直的,也全都是有寬度的,而且歪曲和有寬度的方式各不相同,於是你就錯誤地以為世界上沒有一個直線的原型。其實,各種不完美的直線都是對理想直線的不同方式的模擬。所以,不同的習慣法只是對存在於各種習慣法內部的、沒有被比較膚淺的觀察家發現的那個理念原型的拙劣模仿。

提出這個理論的時候就是文明的最高峰。提出這個理論的知識分子群體不可避免地忍不住要用萬民法來取代各地的習慣法。不是取代我們希臘人、英國人或者法國人那種已經很了不起的、很接近于自然法真理的習慣法,而是取代比如說印度人、土耳其人或者吃人肉的阿茲特克人的那些很明顯是很糟糕的習慣法。

這樣做就必然會引出第三種被蘇格拉底本人和他的大多數弟子認為是不可接受、但是也是一派觀點的學說:什麼是法律?法律就是強者的意志。弱者由於害怕不服從法律將會遭到比服從法律更大的損失,心不甘情不願地服從了強者製造的法律。


\section{}




有很多人以為,文明和溫和的司法是近代以來的產物,古代的司法是極其殘酷的。其實不是這樣。什麼人能夠司法殘酷呢?就是古代的絕對君主國的司法是殘酷的,教會法庭的司法也是比較殘酷的。因為它的法庭相對於訴訟當事人來說是掌握了絕對權力的,它可以用科學方法去探究真相。刑訊逼供實際上是探究真相的科學方法之一,它本身又代表了聰明的知識分子想要從不可知的表象中間探究事實的一種方式。大家不要以為刑訊逼供就是胡亂亂打人,不是的,它也是,像宗教裁判所搞的刑訊逼供,也是有科學程序的,是由精通羅馬法的專家主持,由善於察言觀色各種跡象的人,觀察各種人在撒謊時的各種不同表現,對於不同階級、不同學問、學問大小不同、宗族習慣不同的各種人有詳密的觀察程序,那都不是胡亂搞的東西。但是即使這樣,其實它製造的冤假錯案反而更多。而普通法法庭是完全不搞這一套的。它所承擔的就是自由人和自由人之間的訴訟。自由人是不能打的,就算有罪了,你頂多是把他殺了算是了事,你不可能打他。

雙方之間的訴訟,與其說是一個搞清真相的過程,不如說是一個政治選擇。誰能打贏官司呢?就是那些能夠爭取最多證人的人。誰能爭取最多的證人呢?那就是平時在鄉裡面名譽比較好或者是動員能力比較強,一般來說一呼百應、大家都肯爭取你的這種人。所以,一次普通法的審判,由陪審團主持的一個審判,它與其說像是我們所想象的那樣,以弄清真相為名的司法活動,倒不如說是衡量雙方政治影響的一次政治較量。一般來說,能夠動員最多的鄰居和父老支持的一方,是肯定贏的,無論你的證據是多是少,證人多,那你就是證據最強的。任何證據都比不上是自由人的證言來得有價值。你也可以說,像這樣的法庭其實有點像競選活動,訴訟雙方就像是兩個候選人在爭奪選票,爭奪到當地選民選票最多的一方,差不多就贏了,不是每一次都絕對贏,但是基本上就贏了。法官的作用是消極的,法官也是跟著陪審員的意見走,陪審員就是鄉里選出來的。誰的票數多誰就贏。這基本上是一個政治活動。

而大陸方面,專治君主和羅馬教皇的法庭,那才是真正的科學機構。在這個科學機構之上,法官享有一切權力,而雙方當事人都屁也不是,法官就像是科學家對待小老鼠一樣對待你:這個事情很有意思,我想把真相搞清楚,我就拿你們這些小白鼠做一做實驗,比如說,把你們的腳放在靴子裡面去壓一壓,看看你的反應,如果你是一個軟弱的、一疼就撒謊的人,你會有一種反應,如果是一個堅強的、能熬的無賴,會有另外一種反應。比如說你是一個教士,是一個很愛面子的人,可能會在某一種刺激之下突然不小心的露出真話來,但如果是一個農民的話,對你用的方法就完全不一樣,農民在另一種情況下,在另一種刺激下,才會突然說出真相來。這整個過程是什麼?就是科學家對小老鼠的科學實驗。刑具是什麼?刑具不是別的,就是科學家在小老鼠身上注射這個注射那個的注射器。這個注射器的目的就是,為了通過各式各樣的刺激,從你身上測出各式各樣的反應,然後他把這些反應加以科學的歸類,從中探尋出真相來。只有居高臨下的法官,擁有各種知識和科學儀器的法官才能探究出真相。你當事人已經是什麼也不是了。

歐洲中世紀行使的就是這兩種法律。後來人們所謂的中世紀殘酷,實際上都是根據教會的法庭和大陸國家絕對君主國的法庭得出來的。它們之所以殘酷,為什麼呢?根本原因不是因為它們不夠科學,恰好相反,是因為它們太科學了。科學家和實驗動物之間是沒有平等可言的。他為了瞭解實驗動物的真實情況,是可以無所不用其極的。但是自由人和自由人之間的裁決是跟決鬥是差不多的,跟競選是差不多的。大不了大家可以你死我活,但是雙方都是有尊嚴有底線的。野蠻的和不文明的手段是根本不可能實施的。這裡面歸根結底是一個政治問題。

要是說判決的準確程度,你也不能說是,普通法的法庭就比羅馬法的法庭來得判決正確。實際上,情況可能恰好相反。因為羅馬法的法庭,它的專家也不是吃白飯的,他們那麼多年上大學去學羅馬法,又有這麼多年的司法經驗,他們就是能行。就像中國古代的縣官一樣,你不要以為他們是胡亂的刑訊逼供,拷打別人的,他們就是有察言觀色的本領,像算命先生那種察言觀色的本領,他們知道不同階級和不同身份的人在什麼情況下會說謊,說謊的時候會有什麼樣的表現,如果你真的是存心正當,而且經驗豐富的話,你是可以用刑訊逼供逼出相當多的真情的。一個人在不挨打的情況下,也就是說沒有在受到強烈刺激的情況下,撒謊比較容易,如果我非常瞭解你的階級、出身和性格,我就會知道你這種人在遭到什麼樣的刺激的情況下會不小心說漏嘴。說白了,我要的是什麼,我就是要你不小心說漏嘴,免得你對我撒謊。為了做到這一點,我要給你各式各樣的刺激,這些刺激當中包括刑訊逼供。不是說我喜歡打人或者幹什麼。

但是如果我跟你是平等的自由人,我做法官也好,做陪審員也好,還是做訴訟當事人也好,跟其他人的關係都是平等的,那麼問題就不在於搞不到真相了,而是,我無論如何不能得罪你。真相不真相不重要,關鍵在於哪一方在當地的社會中,在當地的社區中是最得人心的。這才是最重要的。小布什是不是政策最英明的人?比克里的政策英明?誰也說不上。但是如果他得到了更多的選票,就算他是個傻瓜,總統也該讓他當。當美國總統的是什麼人呢?他不是政策最英明的人,不是學問最大的人,就是最得人心的人。在普通法的陪審團中間打贏官司的是什麼人呢?不是最聰明的人,不是證據最確鑿的人,不是最瞭解真相的人,也不是最有理的人,就是最能得人心的人。律師如果善於蠱惑,像辛普森案件那樣,能夠把他的當事人變得比較得人心的話,那麼即使他沒理,他也能打贏官司;反過來,即使你有理,但是如果你不得人心的話,你也打不贏官司。這就是普通法。

你可以說它是,在司法制度中間,隱藏了極其原始的民主,但是我們不要像現在的人這樣,賦予民主太多的意義。現在的人一說到民主,就說是,民主同時代表很多意義,比如說是,公正,自由,平等,其他各種意義都加進去了。原始的民主就是民主。民主是什麼?民主就是能夠爭取多數人的歡心,就是民主。爭取多數人的歡心,那不見得是正義的,多數人不見得是最聰明或者最正義的。普通法所體現的民主,就是這種原始意義上的民主。它經常是不公正的,經常是不聰明的,而且永遠都是不科學的。但它非常有效的阻止了任何超越自由人身份平等之上的武斷權力在英格蘭王國生成。這個武斷權力一旦生成了以後,你再想把它去掉的話,恐怕就非得要有成本很高的暴力流血革命才能夠做到。這條路經是漸漸形成的。我們剛才講到,亨利二世起了很大的作用,但是原因不是在於他熱愛人民,或者是熱愛民主自由,而就是因為他出於技術性的理由想多撈幾個錢,特別是想從其他法庭,特別是從他最討厭的教會法庭身上,把這些錢和權力擠過來,而他又沒有足夠的力量去以其他方式做到這一點。

普通法在以後的幾百年之中,形成幾百年的過程中間,遭到了來自羅馬法的侵襲。這一點不是普通法單獨的特點,所有的日耳曼習慣法都面臨著同樣的問題。從十一世紀到十六世紀之間,復興的羅馬法對各邦的習慣法都構成了強有力的挑戰。任何人都可以看出,晚期羅馬帝國形成的法律高度規範和理性,而日耳曼習慣法是零星的、散碎的、不成系統的。尤其是對於知識分子和有知識分子性格的人來說,接受羅馬法是一個難以抗拒的誘惑。對於想要擴大權力的君主來說,接受羅馬法更是一個難以抗拒的誘惑,因為羅馬法對君主的地位是看得很高的。而日耳曼習慣法當中,要麼給君主的權力跟給原始部落酋長的權力差不多,也就是說基本上是一切都要按習慣法辦。而羅馬帝國晚期的皇帝是享有極大的武斷權力的。所以,國王和知識分子其實都是愛羅馬法超過愛習慣法的。

在十一世紀以前,我們可以說是,習慣法的中心其實不在英格蘭,更多的是在日耳曼。《撒克遜法鑒》產生於日耳曼,而不是產生於英格蘭,就很能說明問題了。但是最後日耳曼完全淪陷了,變成羅馬法的天下;而英格蘭卻變成了習慣法最後的據點。這件事情應該說有很多可能的原因,是由於多個不同因素同時作用於同一個歷史現象的綜合結果。其中比較合理的因素之一就是,亨利二世的改革使普通法提早成熟了。亨利二世按照我們剛才說的那種做法進行改革,他要用英格蘭王國的習俗構成的法律去跟教會實施的羅馬法相競爭,而且他要說他的法庭判案更公正更迅速,各種各樣好,讓羅馬教皇和貝克特大主教的法庭黯然失色。所以,他必須很爭氣,把原先零散的法庭多多少少規範起來,同時也形成了像世紀法庭、巡回法庭這樣比較規範的司法體系。這些司法體系,當初實行的主要動機就是,盡可能的把本來很散的王國本身的習慣法歸攏起來,構成一種跟嚴整有序的羅馬法競爭的體系。在他這個改革的基礎之上,才能出現布拉克頓這樣的法學家,把普通法也當成是一個完整的體系來研究。

這樣,在羅馬法來到英格蘭的時候,它面臨的就不是一些草昧的、純粹的習慣,而是面臨著一群訓練有素的普通法法學家構成的既得利益階層,還有一個雖然仍然是更自由更靈活,但是本身也有一定理論基礎的普通法體系。而在日耳曼,它一方面沒有這樣的一個律師階層,另一方面,它也沒有這個有意識的整頓普通法來跟羅馬法相競爭的動機在裡面。像布拉克頓這個人,還有後來的福蒂斯丘這些人,之所以能夠名垂青史,很大程度就是因為,他們在普通法和羅馬法競爭的過程中,通過為普通法說話,整理普通法,規範普通法,擊退了羅馬法對普通法的進攻。

照現在的記錄來看,就是,蘭開斯特王朝時期,牛津和劍橋兩校,每年都有超過三千的貴族子弟在那裡學習英格蘭的普通法。三千在現在看來是個非常小的數據,但在中世紀是一個極大的數據。我們想想,孔門子弟只有三千。東漢時代,規模相當於整個歐洲總和的大漢帝國,在帝國懸令設立太學生,開始也只有幾千人,最多的時候也只有三萬人。英格蘭王國頂多是相當於漢朝的一個郡,或者是春秋時期的一個諸侯國,它在中世紀這樣草昧的時代,每年都有三千貴族子弟學習普通法。你再比較一下當時的博洛尼亞,意大利各地。意大利各地學習羅馬法的人數,一般來說,普通的一個城邦,甚至像羅馬這樣的大城市,通常幾十個人幾百個人。英格蘭這樣一個邊地的小國,長年累月的,每年都有三千人在這兒學習法律,這就是一個非常驚人的數據了。儘管大陸各國在上層建築這方面通常是比英格蘭王國更先進更複雜的,但是在律師這一方面是個例外。英格蘭王國自古以來就是出律師的地方,它出的律師非常之多,構成了基層政治精英的主力。基本上,蘭開斯特時代的英格蘭王國,是屬於那種每一個郡都有自己的律師團體的一個國家。這一點,不要說歐洲大陸,甚至連最發達的意大利城邦都做不到。

Get Zhongjing Liu | 劉仲敬’s stories in your inbox
Join Medium for free to get updates from this writer.

Enter your email
Subscribe
尤其重要的是,每年培養出來的幾千個法學生,將來就會變成各郡各地的精英階級。他們通過普通法,壟斷了當地的利源。因此,他們有巨大的既得利益,絕對不高興讓羅馬法把普通法的位置侵佔了。當普通法面對挑戰的時候,這些有知識有教養的人,自然而然的會從英格蘭王國傳統習俗中總結出各種各樣的理論,而且把這些理論搞的跟羅馬法學家的理論一樣精緻,這樣就足以抵御羅馬法學家的進攻了;在國王企圖利用羅馬法擴權的時候,這些精英,他們自動的會跟地方精英結合起來,以普通法為保障地方權利和普通的權利工具,反對國王和他周圍的宮廷大臣。這兩個過程決定了普通法在英格蘭的壟斷地位。越往後,這種壟斷地位越會形成路徑依賴,國王或者是教會或者任何其他的人,企圖引進羅馬法,在這個路徑依賴已經形成的過程中間,變得只是孤立而僥倖的襲擊,始終是成不了大氣候的。

而在德國,這些因素統統不存在。可以說,德國維護習慣法的是什麼呢?是他們的騎士、市民,這些人全都是法學方面的外行,他們一再通過國會,通過決議說是,我們日耳曼人不用羅馬那些法律,我們要堅持我們自古以來腓特烈大王時代就已經使用的古老的習慣法,絕對不要引進那些羅馬法。這樣的決定一次又一次通過。可以說,他們通過這個決定的次數,恐怕比起英格蘭國會的牛津條例和類似的條例還要多得多。但是效果卻是很差的。通過的這些決議,基本上是形同虛設,羅馬法學家迅速的搶佔了生態位,跟各邦的專治君主形成聯盟,侵蝕了日耳曼各邦國的傳統自由。原因在哪裡?我想一個重要的原因就在於,提出這些意見的是誰呢?他們是騎士、商人、市民還有自由農民,他們中間沒有法學家。日耳曼儘管產生了《撒克遜法鑒》這樣的偉大著作,但它沒有產生出英格蘭牛津劍橋兩校培養出來的普通法學家這樣一個有系統的法學家體制。

法學家在英格蘭,它是一個階級,而且是一個已經形成了壟斷性、在教育界政界和地方各界中間盤根錯節的一個師徒傳授的一個體系。任何人如果想要混得出點頭,都要到四大法學院去混一段時間,跟原有的普通法律師培養好關係網才能混得下來。而在日耳曼呢,那些市民也好,騎士也好,農民也好,他們沒有這樣的關係網,他們沒有這樣的盤根錯節的律師階層。羅馬法學家一旦掌握了君主和他的宮廷,接下來就可以勢如破竹的把羅馬法推行下來。君主喜歡羅馬法和隨之而來的巨大君主權力,每一次都通過引用羅馬法,提高君主地位,而把農民貶低成農奴。在中世紀早期,其實日耳曼的自由民傳統,可能比英格蘭還要更強一些。但是到中世紀快要結束的時候,日耳曼農民已經處在嚴重的依附狀態,在波蘭那些地方甚至出現二度農奴化。而英格蘭,不僅原有的自由民仍然是自由民,而且農奴已經全部解放了。

普通法學者通過封建慣例的堅持,使英格蘭的自由民階級逃過了地理大發現帶來的通貨膨脹,而這個通貨膨脹把英格蘭的貴族和王室給坑慘了。在日耳曼則發生了相反的過程,通貨膨脹的壓力迅速壓到了市民和自由農身上,使他們大批的破產,而他們的破產造成了領主權力的急劇擴張,使得原先在中世紀早期非常繁榮昌盛的日耳曼城市一個一個的走向沒落,最後變成地方性專制君主的附庸。原先曾經強大的自由農和市民,經過幾次不成功的內戰以後,喪失了他們原有的自由,變成了普魯士、撒克遜或者是巴伐利亞這些君主的附庸。而與此同時,本來很軟弱的英格蘭的農民和市民,卻在普通法的保護之下越來越強大起來。

這個保護是非常具體的,它用一個簡單的方法就可以做到。你能不能夠堅持中世紀早期留下來的那些慣例,對你來說是一個直接關係到錢袋子的問題,不是個抽象的理論問題。因為在美洲金銀進來的情況下,普遍出現的情況就是,物價急劇上漲。急劇上漲的壓力壓到誰的頭上去是很成問題的事情。如果我們按照中世紀的老規矩辦的話,那麼按照過去物價低廉的情況下,你交納的費用是非常之少的。比如說是,我是德文郡的一個副本產業所有人,照現在的話來說,我是一個租地人。但是中世紀的租地人可不像現在的租房子的人那樣沒有保障,只要業主高興就可以趕你走。當地的租地的人,也就是說,副本他們,當地教區的名冊上面有他祖父曾祖父的名字,那麼他世世代代都在這裡,任何人不能把他趕走。這片土地,只要是你在威廉那個時代或者是阿爾弗雷德那個時代租給了他的高高高高高高高高高曾祖父,而且在教區上留下名字,他就憑這上面寫下來的名字,他世世代代都能賴在這兒不走。領主沒法把他趕走,地主沒法把他趕走,任何人都沒法把他趕走,他的權利比起資本主義社會下的租房客或者是無產階級是要強得多的,他是當地社區有固定地位的成員。

很好,然後他就會拿出他古老的古老的古老的抄本,來證明,在偉大的阿爾弗雷德時代,或者在愛德華一世時代,我們跟你早就商量好了,我的義務就是,給伯爵大人兩袋小麥,然後伯爵大人自己派你的管家來取小麥。如果伯爵大人要求我把小麥送到你們的倉庫里,對不起,這不符合我們原有的規定。我們已經規定好了,我的義務是,勻上兩袋小麥給你,再幫你打四十天仗,然後再花一個星期時間幫你修水渠。這些東西都已經寫得好好的了,你再讓我替你搬運你的小麥,這就是在已經寫定的條件之上,另外給我加了額外的義務。老子不幹。我已經給你修了一星期水渠了,再也不會替你搬麥子。搬麥子是你自己的事情,你不來搬我就自己留著吃了。你如果不高興的話,那我們到莊園法庭上去打官司。如果普通法是足夠強大,那麼,古老習俗就會勝利。到頭來的結果就是,伯爵大人還是得自己去搬麥子。反正他的佃農就是不給你搬。

如果說是,物價上漲了,以前這兩袋麥子代表的錢是足夠你生活,但是,現在物價上漲了七八倍,領主也想給自己多撈點錢,但是也沒有辦法。你不能說是,現在物價上漲了五倍,所以你以前交兩袋麥子,我就收你十袋麥子。這個不行。這是違反古老習慣的。莊園法庭不會支持你。莊園法庭的陪審員都是當地的父老鄉親,當地的父老鄉親之間的看法跟我是一致的,他們不高興搞出一個先例來,讓我多交八袋小麥,然後以後他們不是也得跟著這個先例多交八袋小麥嗎?萬萬不行。僅僅憑打官司這一項,我讓你這個稅收永遠漲不上去。稅收永遠漲不上去,但是國王或者貴族並不能少花錢,因為戰爭已經升級了,過去用民兵,免費的弓箭手 — — 像羅賓漢那種人,這種人平時在家鄉的農閒時間就經常去射箭打獵,所以免費的學到一手好箭法,打仗的時候這些人就跟著國王到法國去,發出一連串亂箭,把法國的騎士從戰馬上射下來 — — 這是很廉價的事情。但是現在弓箭手不流行了,改用火槍手了。火槍是要花錢買的。這筆錢誰出?根據剛才的邏輯,你只要打不散普通法的律師和莊園法庭的話,你別想從農民身上搞到這筆錢。儘管農民通過農業技術的改善,原先我收十袋小麥的時候約定給領主兩袋小麥,但是現在我明明已經收到一百袋小麥了,但是我還是只給你兩袋小麥,就因為莊園法庭根據古老的慣例,只讓我交這麼多小麥。以前我的利潤是八袋小麥,現在我的利潤漲到九十八袋小麥,但是這九十八袋小麥全歸我這個富農本身。

於是就通過這樣的程序,原先出身微賤的,像帕斯頓家族這種,其實說白了就是佃農,甚至可能是農奴的人,通過把他原來的八袋小麥變成九十八袋小麥、交納給領主的小麥還始終是兩袋這個過程,漸漸的變成了富農,然後漸漸的變成了鄉紳。最後,伊麗莎白朝以後的鄉紳,實際上已經不是原有的貴族,而是這批新興的暴發農民。他們依靠習慣法的保護,取得了他們新的地位。而原有的貴族因為得不到增加他新有的收入,卻必須支付火槍和更多的開支,漸漸趨於破產。而英國內戰的爆發,實際上是這個破產的一個總清算。最後的結果是,薔薇戰爭和英國內戰以後,傳統的封建貴族基本上被替代了,新的紳士實際上是來源於原先受普通法保護下的有產階級。他們形成的有產階級構成了英國政治體系的基礎,是他們提出了沒有代表權就不納稅的邏輯,也是他們製造了國會政治。是他們開創的殖民地,通過殖民地和特許權,把原有的普通法普及到全世界。

大家要知道,殖民主義這件事情也是有不同類型的。像法國和西班牙,它的殖民主義就是什麼?本土行政官員制度的擴大。我西班牙王國已經消滅了國內各等級、各自治市,把我的行政官吏派到塞維利亞和馬拉加,然後我征服美洲以後怎麼樣呢,我派總督到秘魯,派總督到智利,到墨西哥和其他地方去,把我的行政官派到這些地方去。法國也是這樣。而英國不一樣,英國還是根據普通法保留的封建遺產,國王是不會派行政官去的,我只會發出特許權,給你一個特許權,你們自己根據這個特許權授予這個法人團體。這個法人團體,馬薩諸塞公司,你的法人地位跟倫敦市和布里斯托爾市是相同的。你拿著這個特許狀,給國王交一點錢,好了,你買到了你的自治權,憑著你的自治權你自己去幹吧,建立城邦也好,開拓殖民地也好,都是你自己的事情。通過這樣的特許權制度,它就把,可以說是古老的日耳曼的封建自由,播種到英格蘭殖民地。它產生的就是一個活的有機體。馬薩諸塞公司或者東印度公司都是活的,馬薩諸塞公司可以發動第二次殖民。比如說像美國現在的緬因州,它就是二次殖民的產物。馬薩諸塞是英國殖民者組成的馬薩諸塞公司的產物,而緬因州是馬薩諸塞殖民者二次殖民,進一步到北方開拓荒野的結果。

它搞出的殖民地,就是中世紀的自治團體。自治團體可以有絲分裂的,我這個自治團體再產生出新的自治團體,然後不斷的向遠方去。比如說來到了遠東,就產生了上海的工部局,產生了香港的納稅人會議,產生了漢口外灘的納稅人會議,產生了新加坡,產生了馬來亞聯合邦。英國殖民就是這樣用自我繁殖的方法產生的。這樣產生的自治團體,至少對於國家是沒有成本的,它是自治團體成員自己籌資開發,像一個資本主義的公司一樣,我們自己股東湊錢開發,開發出來的利潤我們股東分。也許要給國王四分之一或者給國王一些其他什麼好處,但是,主要部分是我們這些股東自己分的。

這是殖民主義的起源,是英格蘭古老自由的起源,也是資本主義的起源。因為Corporation這個詞有好幾種意義,它的意義包括:法人團體,自治市鎮,殖民地公司。也就是說,我們自己組織一個棉花貿易公司或者是計算機出口公司,在法律上講,跟倫敦市、布里斯托爾市、馬薩諸塞公司、哈德遜灣公司、東印度公司,是同一個性質的。現代資本主義,不是別的,它就是倫敦市和馬薩諸塞公司和東印度公司的階級兄弟。它們都是同一個封建特許權的產物,繼承的都是中世紀英國的古老自由。

自治團體這件事情,是英格蘭憲制的結果,通過英格蘭的殖民主義,最終擴充到全世界,構成了現在世界秩序的重要組成部分。我們可以下一個不是絕對正確、包括許多局部的例外、但是總體上和大體上正確的判斷,就是說:現在這個世界,如果是自由的國家,那麼它十之八九,直接或間接的,不是模仿英格蘭王國,就是來自於英格蘭的殖民主義輸出 — — 英格蘭的殖民主義輸出,就是我剛剛講的那個自治市政體系;凡是不來自於這個體系的所有國家和地區,建立的都是行政官專制的武斷體系。可以說,現在的體系,儘管英國已經被美國取代了,它的核心的制度是什麼?仍然是英格蘭的古老自由。美國的政治體系比英國更好的體現了英格蘭的傳統自由,因為英格蘭在第一次世界大戰以後,多多少少受到歐洲大陸的影響。

而家成個世界格局個核心,睇落去好似新鮮,其實骨子裏都係舊酒裝新瓶。簡單四隻字——自治社團。呢樣嘢,老到出汁。老過阿爾弗雷德大王,老到可以一路追返去日耳曼森林。近代美國堪薩斯州嗰啲陪審團,比起英格蘭嘅樞密院,甚至世上大部分國家嘅法律,都仲要似返日耳曼森林入面啲長老圍埋審案嗰種局。呢個,就係現代世界最核心嘅「遊戲規則」。

至於喺呢套規則以外仲有啲例子,好似中國噉,背住布爾什維克嘅尾巴,又繼承咗東方專制嘅命根。又例如法蘭西、西班牙,仲攬住絕對君主制嗰條舊繩。但講到底,呢啲都係被逼蹲喺世界秩序邊皮,食煙火。真正坐正喺中央嘅,就係由日耳曼、英格蘭一路傳落去到美利堅,呢條血脈打出嚟嘅自治市政、法人團體、自由組織——砌成嘅自由帝國。今日我哋身處嘅世界,依然係喺佢哋班友嘅治下。凡係唔啱規矩、唔入局嘅,一律踢去邊皮角落度飲西北風。

好啦,多謝晒,講到呢度收工。





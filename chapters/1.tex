\chapter{What is English Rationality?}

\separator

English Rationality is the rationality embedded in the English language. it is the native rationality of English. it is begotten by English, and is of English in the sense that its colour, temperament, and mettle of the English language. 

\separator
English Rationality, is first and foremost, a rationality. What is rationality? It is the quality or substance or state of being rational. If "being rational" is a thing, that thing is rationality. What is a "rationality"? Rationality, prefixed by with an indefinite article "a" implicitly posits that there are multiple and other rationalities, i.e. there are multiple ways of being rational. English Rationality is merely one form of rationality, and it is the English one. 

\separator

English Rationalituy is a rationalituy, and what is a rationality? Rationality comes in many layyers and degrees of sophsitication. But from a bird's eye view, rationality, is the setate of being able to put forth propositions, intelligently argue for them, ask "what if", contemplate what could or would have been, imagine the otherwise, and to make definitive conclusions after such exercises. 

\separator

English Rationality, is the ability to eneterain an idea without accepting it. it is the ability and the habit to consdier ideas and propositions like objects of interest in the head. 

\separator

English Rationality is that which can tell us what is reasonable, and what is not.
What is reasonable, is not necessarily what is justified, legitimate, or sound. It is something slightly weaker, thought simulatenously stronger and sounder than  nonsense or the absurd. The justified is what is already reasoned, therefore its reasonability is already established. The justified observes and recognizes events and facts, faithfully and accurately and describe them. 


What is "reasonable"  therefore only by defacturied convenience, just like how all women are girls. But the strict category what is reasonable, unless to the class of ideas and propsotions that to a person of sound mind i.e. endowed with the operationsal software of English Rationalitym would find to be jstufied, in a vague s and mecurial terms. He could see how the reasonable might be reasoned to establish itself, and therefore be reasoned when justified to do , even thought an actual application of the reasoning process might willingly yield a verdict in the negative. The reasoanble proposition is one that could be reasoned, and its status of freasoanble, is a state of onterm likelihood. 

Reasonability lends reasonability, reasonability is preserved between ideas that entail and ideas that are enetailed, just like truth is preserved in between statements. And to prove a statement, being proved. 


And so, in reasonability, one yield and inspiring interlocking web of reasonable ideas, each supporting others, such that one have fashioned a Grand Republic of reasonabilities. To carry this republic and to uphold, to operate it like a juror operates a jury, is to be in possession of English Rationality.
\separator

English Rationality, is therefore, very naturally, "Common Sense". 

"Reasonability" as the connector between propsoitions as aforementioned, is perhaps an inaccurate protrayl of this logical substance. That's to say, to characterize a proposition as reasonable, or to say "such and such is reasonable" if you accept so-and-so" means  there is a way to employ reason toand to explicate a line of reasoning from such-and-such so that so-and-so is demonstrated", is inaccurate. When one says "such-and-such is reasonable", what one is really saying that by the powers of common reason, and by adequate judgements, such-and-such is admissable as truth". In some cases, it might indeed turn out that there is an argument for such-and-such, but itw would be wrong to claim a statement is reasonably, if and only if, you can reason out the statement. When jduges and schoalrs and statements and phiklosophers claim pocilies and propositions to be reasonable, what they are reaally asserting, that they find the proposition in question to be admissiable, and it is admissiable because it was a reasonable person, a person of sound mind, a person of intellectual and adjudicational sincerity and integrity, that made this call - a call a judgement, an assessment - that is fundmanetllay, if one gets tot he bottom of it, ultimately arbitrary.

\separator


 English Rationality, is the language and mode of thinking of Hume, Locke, Mill, Benetham, Denning, Lord Wilson, Lord Patten, Churchill, Russell, Stephen Fry, Richard Dawkins, Oscar Wilde, Christopher Hitchens, Derek Parfit, and Alex O Connell. It is language precisely used, to articulate ideas reasonably well formed. 

\separator. 

What is English Rationality? It is something so embedded in English, and gives English such rational power, that oftentimes when one want to demand another to speak plainly so one may understand, one might simply make the demand "Speak English". "Speak English!" is a demand to speak reasonably, rationality, and understandably. "I can't understand this, please speak English" is a polite request to make complex ideas understandable. A chief executive running a financial empire, requesting his team to "Speak English", is to request for accommodation for lower intellectual bandwidth and horsepower. English Rationality, is that gem in the English language that enables this.

\separator
English Rationality has begotten Legalese. Legalese, is not English. "The contract is in Legalese, sir, let me put it in English for you." In that sense, Legalese is not English. But obviously, it the English language. It is the English language in a particular register, used in a particular profession. It is clearly borne of the English language, and is trained to be performant and capable in handling legal action and judisprudential questions. It is capable of doing so because it contains a particular mode of highly specified form of English Rationality. 


\separator

English Rationality, is begotten first and foremost, an awareness of syntax,  an intuitive unazed grasp of grammar, so there is a deletection of whether a sensetence has or has not any sense. There are languages in want of such a sense - a sense to detect grammatically right sentences from wrong ones. 

There are languages spoken by people with no robust sense of grammatical propriety or validity, and so sentences that are ill-formed might still be bubbling around ocassionally in conversation, without inducing judgement or stigma. English has this sense French has this sense. Japanese has it. And so do the rest of the world's great classical languages, such as Latin, Greek, Hebrew, and Sanskrit. But not classical Chinese. And for Mandarin Chinese, it's there, but significantly weaker than English. Cantonese is even weaker. What does a alagnauge look like when its own grasp of grammtical propriety and syntaxical validity is in want of lartfication? How does a lagnague look like, when its own speakers do not really  know the grammatical constraints of the language that they speak? Grammatical sure - this what underGrammatical sense - this is what underforms all operations of rationality that come after. Grammatical sense, is the intuition of whether an utterance is grammatcially wellformed. Obviously what is grammatically valid is not delineated by dictionaries, nor can be prescribed by the dictates of men, but are undescribable schemas that ebb and flow, as the mores and customs of the language's speakers themselves change. Segments and subsets may be framed in descriptive rules, but alas, tis impossible to ecludidate it all exhaustively, across space and across time. If you try to enumerate them all, in a neat and tidy fashion, you will find yourself forced to exclude the untidy exceptions, and ensnared to assume an authoritarian position to declare and proclaim what is regular and what is deviant, separating what are all naturally emerging customs into the licensesed and the outlawed. At end of such an exercise, you'd find yourself either with formal languages like Aristotleian logic, or a highly regimented and precise quasi-constructed language like Sanskrit a la Ashtadhyayi. Either product has a natural trajectory by the natural uses of language by man to intermingle with the languages, and ttherefore grammars and rationalities, of the language whence they came - and so infect them with their prescriptively purified mechanics. 


Butstill, gramatically sense, epheremal aand as democratically annoitned in legitimacy as it is hazy and indeterminant, i nevertheless empowers it speakers to differentiate a well formed fomrula from a malformed formula. And rationality, the modes of grammatical sense comes in grades. Some are more mature, and some can process more complex sentences with more complex grammar, wheteras others cannot. All langauges and their speakers are obviously enclosed with a minimally viable grammatical sense - minimally viable in the sens e that it allows the speaker to form syntaxically understandable and comprehensible utterances  aderquate for their linguistic lives. But there is absolutely no reason for us to believe that grammatical sense of different langagues must be equal in sophistication or power. It is not. 

this sense of the grammatically correct, would evolve as much as it can devolve. 


Once we have a sense of what are well formed sentences, then we can speak of whether they have sense, whether they are meaningful. Whether they are actually talkinag aobut something. From here, the groundwork for argumentation is founded, for mif meaning can be discerned, then whether the meaning bluurbed and uttered conrresponds to truth, can now be considsered, nalayzed, and eappreciated. Such an action, if done wit multiple pasties, with multiple meanings, contending for truth, is whawt we call a debate. Debate invovles the often rapid and emotional as well as intllelectual delivery or argumentative payload. it involves finding many meanings altoghether. it is the process that humans are tested if they appreciate certain pattens of reasoning. 

In argumentation, an argument is formed. Whether the argument is accepted, depends on whether it is acceptable per some framework or system of rules assuming the partier involved have ht esufficient brainpower to make the assessment by applying the rules - what argument is or most allowed ins of course precisely what logic prescribes. The application of logic, to derive to compute to exteract conclusions is of course what logical reasoning is. But it'd be a mistake to think that it is the whole of what there is to say about logic. Logic can only determine what is acceptable. An argument may be deemed acceptable, if you subscribe to a logic, but whether the logical framework is applpicable is also adequate or even reelevant, or appropriate, that is a question that the logical framework itself cannot rule on. 

An argument may be acceptable per a logic, but whether that logic is the appropriate logic to make this determination, is ahigher and external question that cannot be decided with the logic itself. 

One might be uncomfortable making this claim, because and therefore there are multiple logics. Therse logics are most probably not organizable into some simple one-in-another hierarchy, where eahc logic is simply a sublogic of the next, nested in each other like russian dolls. No they are fudnaemntally different logics, incompatible and mutually exclusive with each other in application and in metaphysical commitments. Why such diversity and adamant personalities? There are different logics because they are formal systems extracted from sentences of different customs and debates of different parliaments. A logic emerges in a custom because of the dynamics of that custom. A logic emerges in a debate of a parliament because of the dynamics of that parliament. Different gardens populate different flowers. Different arguments, different logics extractable. There is no reason why a transcendental logic extracted from a family of transcendental arguments, would be compatible with the simple aristotlean syllogism. There is no reason to believe that the if of "what could have been" would have the exact behaviour and operational intricacy of the if of "if-this-then-that".

English rationality is these logics, extracted, formalised, from these clubs of arguments, and the corpuses of arguments that lend plpasuaiblity and admissiability to these logics, and the intution necessary to recognize and choose between them. 

English Rationality, is not just the fruits of logics borned of the garden that grows such logics. It is also the sense whether a logic is to be used, just like whether a fruit should or should not be used in a given pie. 

\separator
English Rationality has endowed the English language with the extraordinary ability to make anything sound reasonable. With English Rationality on your right hand, even the smallest of smallfolk can sound like a seasoned statesman or an average philosopher. It is for this reason, that English Rationality - or rather - the logical and rhetorical component of the English Rationality machine, without a significant and robust body of obstinante feelings, and a canontuitituy that is bound together by common blood and custom, risks becomign the source of any astoudning dielogy. English rationality, when commended by folks other than the Englishman (the Brits), will gradually take on the colour and cahracter of its adopted user, gradually accommodating these alien values and customs - no so different from the Englishmen who first resided in Bombay and started to take on Hindoo wives, or how Lee Kwan Yew's invention of Asian Values in the form of free press suppression, slanted playing field in elections, and the disregarding of rights private, natural, and eternal in Speech has seeped into the common witticisms and idioms of the Singaporean populace - or alas - just the simple bastardisation of the Common Law in Hong Kong, with doctrines forcibly impregnated by the Chinese imperial drive - or the modern leftist fanatic to recompose the definition of womanhood, violence, or equity.


\separator

What is English Rationality? English rationality and its impeccable ability to make dissections and analysis, is what gives parliamentarianism its lifeforce and energy, and what gives credibility and legitiamcy to Common Law. English Rationality's ability to justify, injects legitiamacy into its arguments, and grants them the title being justified. Those who want to challenge such grants, they only need to speak English and check the grant themselves, by assuming the position of the grantor, consider the argument in English, and then judge the arguemnt for itself. English Rationality is what guides one to do this job well.
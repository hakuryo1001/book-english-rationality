\chapter{English Rationality is eating itself}

Harry Lee Kwan Yew of course mastered English Rationality.


It is now common for people to think that thinking and intellectualism is all about coherence, and that empirical judgements are to be eschewed. This is essentially equivalent to only discussing the consistency of propositions but never their own truth value. That has become a task relegated to the realm of emotions, and therefore private and respectable regardless of content. The English Rationality does this because they want to ground their paradigm out of nothing, so that it can become self evident, a tautology. But the problem of course, is that such activity is impossible. We live, in a contingent world. That contingency might be necessary, but we not living in a necessity. 


Only the English speaking fool could believe this. The English speaker masks his ignorance and stupidity by the sheer rationality of the English language. 



This is what I’ve been saying for the past year. The English speaking world is unfolding, twisted into a knot in its own body by its own impeccable rationality. As far as I see, its demise is inevitable. The hitherto unstoppable expansion of the English empire will now halt, and shall enter a long meandering age of managed decline. What will rise to supplant her? None under heaven I answer. 


It’s as if Britain had turned herself into one of Lewis Carroll’s novels. “We’re all mad here,” proclaimed the nation that gave birth to the rule of law, analytic philosophy, sublime Rationality, and the English language. 